\documentclass[aspectratio=169]{beamer}

% Custom UU AIF Beamer style
\usepackage{uu-aif-beamer}

% Packages
\usepackage{amsmath}
\usepackage{amssymb}
\usepackage{graphicx}
\usepackage{booktabs}
\usepackage{hyperref}
% \hypersetup{
%     colorlinks=true,
%     linkcolor=black,
%     urlcolor=blue,
%     citecolor=blue
% }

\begin{document}

% Custom title slide
{
\setbeamercolor{background canvas}{bg=darkbg}
\setbeamertemplate{footline}{}
\begin{frame}[plain]
  \begin{tikzpicture}[remember picture, overlay]
    \fill[darkbg] (current page.south west) rectangle (current page.north east);
    \node[anchor=center, opacity=0.05] at (current page.center) {
      \includegraphics[width=\paperwidth, height=\paperheight]{\uubg{front_page_chip.png}}
    };

    % Custom multi-line title
    \node[anchor=west, align=left] at ([xshift=-5.5cm, yshift=1.5cm]current page.center) {
      \color{uuyellow}
      {\fontsize{32}{38}\selectfont Algorithms in Finance}
    };
    \node[anchor=west, align=left] at ([xshift=-5.5cm, yshift=0.5cm]current page.center) {
      \color{uuyellow}
      {\fontsize{20}{24}\selectfont Volatility Modelling and Forecasting}
    };

    % Logo
    \node[anchor=south west] at ([xshift=0.8cm, yshift=0cm]current page.south west) {
      \includegraphics[height=2cm]{\uulogo{UU_logo_2021_EN_WHITE.png}}
    };

    % Date
    \node[anchor=south east] at ([xshift=-0.8cm, yshift=0.8cm]current page.south east) {
      \color{uuyellow}
      % {\small \today}
      {\small February 2026}
    };
  \end{tikzpicture}
\end{frame}
}

% Outline
\begin{frame}[nologo]{Outline}

\vspace{0.2cm}
\textbf{\textcolor{blue}{Today's Lecture:}}
\begin{itemize}
    \item Recap: Volatility Foundations
    \item Stylized Facts of Volatility
\end{itemize}

\vspace{0.4cm}
\textbf{\textcolor{orange}{Workshop Session:}}
\begin{itemize}
    \item Forecasting Volatility
    \item Forward-Looking Volatility as a Trading Signal
    \item Conclusion
\end{itemize}

\end{frame}

% =====================================================
% SECTION 1: RECAP - FOUNDATIONS
% =====================================================
\section{Recap: Volatility Foundations}

\begin{frame}[nologo]{What We Already Know}
Building on Hull's \textit{Risk Management and Financial Institutions}:
\begin{itemize}
    \item Exponentially Weighted Moving Average (EWMA)
    \item GARCH(1,1) models
    \item Maximum Likelihood Estimation
    \item Applications: Value-at-Risk (VaR)
\end{itemize}
\vspace{0.3cm}
\textbf{Goal:} Quick review before advancing to more complex models, forecasting, and trading applications
\end{frame}

\begin{frame}[nologo]{Returns and Their Distribution}
\textbf{Price to Return Transformation:}
\begin{align*}
r_t &= \log P_t - \log P_{t-1}
\end{align*}

\textbf{Return Decomposition:}
\begin{align*}
r_t &= \sigma_t z_t
\end{align*}

\textbf{Conditional Variance:}
\begin{align*}
\sigma_t^2 &= \mathbb{V}\left[r_t|\Omega_{t-1}\right]
\end{align*}

where:
\begin{itemize}
    \item $z_t \sim \mathcal{N}(0,1)$ i.i.d for all $t=1,\ldots,n$
    \item $\Omega_{t-1}$ denotes the information set (history of the time series)
    \item $\sigma_t^2$ is the conditional variance (time-varying volatility)
\end{itemize}

\vspace{0.3cm}
\textbf{Key Insight:} Returns are unpredictable, but \textit{volatility} is predictable!
\end{frame}

\begin{frame}[nologo]{EWMA: Exponentially Weighted Moving Average}
\textbf{Basic Idea:} Recent observations matter more than distant ones

\begin{block}{EWMA Formula}
$$\sigma_t^2 = \lambda \sigma_{t-1}^2 + (1-\lambda) r_{t-1}^2$$
\end{block}

where:
\begin{itemize}
    \item $\sigma_t^2$ is the variance forecast for time $t$
    \item $r_{t-1}$ is the return at time $t-1$
    \item $\lambda$ is a decay factor (typically 0.94 for daily data, RiskMetrics)
\end{itemize}

\vspace{0.3cm}
\textbf{Key Properties:}
\begin{itemize}
    \item Simple to implement
    \item No parameters to estimate (if $\lambda$ is fixed)
    \item Reacts quickly to volatility changes
\end{itemize}
\end{frame}

\begin{frame}[nologo]{GARCH(1,1): Generalized Autoregressive Conditional Heteroskedasticity}
\textbf{Motivation:} Allow data to determine optimal weights

\textbf{GARCH(1,1)} --- \href{https://www.sciencedirect.com/science/article/pii/0304407686900631}{Bollerslev (1986)}:
$$\sigma_t^2 = \omega + \alpha r_{t-1}^2 + \beta \sigma_{t-1}^2$$

where:
\begin{itemize}
    \item $\omega$ is the long-run variance component (constant)
    \item $\alpha$ is the reaction to new information (ARCH term)
    \item $\beta$ is the persistence of volatility (GARCH term)
\end{itemize}

\vspace{0.3cm}
\textbf{Constraints:}
\begin{itemize}
    \item $\omega > 0$, $\alpha \geq 0$, $\beta \geq 0$ for non-negativity
    \item $\alpha + \beta < 1$ for stationarity
    \item Long-run variance: $\sigma_L^2 = \frac{\omega}{1-\alpha-\beta}$
\end{itemize}
\end{frame}

\begin{frame}[nologo]{Maximum Likelihood Estimation (MLE)}
\textbf{Question:} How do we estimate $\omega$, $\alpha$, $\beta$?

\textbf{Answer:} Maximum Likelihood Estimation

\begin{block}{Log-Likelihood Function}
$$\mathcal{L} = -\frac{1}{2}\sum_{t=1}^{T} \left[\ln(2\pi) + \ln(\sigma_t^2) + \frac{r_t^2}{\sigma_t^2}\right]$$
\end{block}

\textbf{Procedure:}
\begin{enumerate}
    \item Start with initial parameter values
    \item Compute $\sigma_t^2$ for all $t$ using GARCH formula
    \item Evaluate log-likelihood
    \item Use optimization algorithm to maximize $\mathcal{L}$
    \item Standard errors from inverse Hessian (if model is correctly specified, see \href{https://doi.org/10.1080/07474939208800229}{Bollerslev \& Wooldridge (1992)})
\end{enumerate}
\end{frame}

\begin{frame}[nologo]{Application: Value-at-Risk (VaR)}
\textbf{Definition:} VaR at confidence level $\alpha$ is the loss level that will not be exceeded with probability $\alpha$
\begin{columns}[T]
\column{0.52\textwidth}


\begin{block}{VaR Calculation with GARCH}
$$\text{VaR}_\alpha = -\mu_t + z_\alpha \cdot \sigma_t$$
\end{block}

where:
\begin{itemize}
    \item $\mu_t$ is expected return (often assumed 0)
    \item $z_\alpha$ is a quantile of the distribution\newline (e.g., -1.645 for 95\% VaR for Standard-Normal Distribution)
    \item $\sigma_t$ is the volatility forecast
\end{itemize}



\column{0.46\textwidth}
\vspace{0.4cm}
\begin{center}
\includegraphics[width=\textwidth]{assets/other/var_distribution.pdf}
\end{center}
\end{columns}

\vspace{0.2cm}
\textbf{Example:} 95\% daily VaR, \$10M portfolio, $\sigma_t = 2\%$:
$$\text{VaR}_{95\%} = 1.645 \times 0.02 \times \$10M = \$329,000$$
\end{frame}


% =====================================================
% SECTION 2: STYLIZED FACTS
% =====================================================
\section{Stylized Facts of Volatility}

\begin{frame}[nologo]{Empirical Observations: Stylized Facts}
Financial returns and its volatility exhibits characteristic patterns (\href{https://www.tandfonline.com/doi/abs/10.1080/713665670}{Cont, 2001}):

\begin{enumerate}
    \item \textbf{Volatility Clustering}: High volatility periods cluster together
    \item \textbf{Mean Reversion}: Volatility tends to revert to long-run average
    \item \textbf{Non-Normality of Returns / Fat Tails}: Return distributions have heavier tails than normal
    \item \textbf{Leverage Effect}: Negative returns $\rightarrow$ higher volatility
\end{enumerate}

\vspace{0.3cm}
\textbf{Additional stylized facts} (covered in appendix):
\begin{itemize}
    \item \textbf{Long Memory}: Volatility autocorrelation decays slowly (hyperbolic)
    \item \textbf{Structural Breaks}: Parameters shift across economic regimes
\end{itemize}

\vspace{0.3cm}
\textbf{Implication:} We need models that capture these features!
\end{frame}

\begin{frame}[nologo]{Stylized Fact 1: Volatility Clustering}
\begin{center}
\textit{"Large changes tend to be followed by large changes, of either sign,\\
and small changes tend to be followed by small changes."}\\
\vspace{0.2cm}
--- \href{https://link.springer.com/chapter/10.1007/978-1-4757-2763-0_14}{Mandelbrot (1963)}
\end{center}

\vspace{-0.2cm}
\textbf{Evidence:}
\begin{itemize}
    \item Visual inspection of return series shows clustering
    \item Significant autocorrelation in $|r_t|$ and $r_t^2$
    \item Little autocorrelation in $r_t$ itself
\end{itemize}

\textbf{Models That Capture This:}
\begin{itemize}
    \item ARCH/GARCH family
    \item HAR family
    \item Stochastic volatility models
\end{itemize}
\end{frame}

\begin{frame}[nologo]{Stylized Fact 1: Volatility Clustering}
\vspace{0.6cm}
\begin{center}
\includegraphics[width=0.95\textwidth]{assets/other/volatility_clustering.pdf}
\end{center}

\end{frame}

\begin{frame}[nologo]{Stylized Fact 2: Mean Reversion}
\textbf{Observation:} Volatility doesn't stay high (or low) forever

\textbf{Mathematical Expression:}
$$E[\sigma_t^2 | \mathcal{F}_{t-1}] \rightarrow \sigma_L^2 \text{ as } t \rightarrow \infty$$

\textbf{In GARCH(1,1):}
\begin{itemize}
    \item Speed of mean reversion controlled by $\alpha + \beta$
    \item If $\alpha + \beta \approx 1$: very slow mean reversion (high persistence)
    \item If $\alpha + \beta \ll 1$: rapid mean reversion
\end{itemize}

\textbf{Practical Importance:}
\begin{itemize}
    \item Long-horizon forecasts converge to long-run variance
    \item Crisis volatility eventually subsides
\end{itemize}
\end{frame}

\begin{frame}[nologo]{Stylized Fact 3: Non-Normality of Returns / Fat Tails}
\textbf{Observation:} Returns have heavier tails than Normal distribution


\begin{center}
\includegraphics[width=0.85\textwidth]{assets/other/fat_tails.pdf}
\end{center}

\end{frame}

\begin{frame}[nologo]{Stylized Fact 3: Non-Normality of Returns / Fat Tails}
\textbf{Observation:} Returns have heavier tails than Normal distribution

\textbf{Standard GARCH:}
$$z_t = \frac{\varepsilon_t}{\sigma_t} \sim N(0,1)$$

\textbf{Problem:} Empirical returns show excess kurtosis

\vspace{0.2cm}
\textbf{Solutions:}

\begin{itemize}
    \item \textbf{Student-t:} $z_t \sim t_\nu$
    \begin{itemize}
        \item Typical: $\nu \approx 5-10$
        \item As $\nu \rightarrow \infty$: $\rightarrow$ Normal
    \end{itemize}

    \item \textbf{GED:} $z_t \sim \text{GED}(\nu)$
    \begin{itemize}
        \item $\nu = 2$: Normal
        \item $\nu < 2$: Fat tails
    \end{itemize}
\end{itemize}


\vspace{0.2cm}
\textbf{Implementation:} Modify log-likelihood function during MLE estimation
\end{frame}

\begin{frame}[nologo]{Stylized Fact 4: Leverage Effect}
\textbf{Asymmetric Response:} Negative returns increase volatility more than positive returns

\textbf{Two Main Explanations:}
\begin{enumerate}
    \item \textbf{Financial Leverage} (\href{https://doi.org/10.1016/0304-405X(82)90018-6}{Christie, 1982}):
    \begin{itemize}
        \item Fixed debt + falling stock price $\rightarrow$ higher debt/equity ratio
        \item Higher leverage $\rightarrow$ higher risk $\rightarrow$ higher volatility
    \end{itemize}

    \item \textbf{Volatility Feedback} (\href{https://doi.org/10.1016/0304-405X(92)90037-X}{Campbell \& Hentschel, 1992}):
    \begin{itemize}
        \item Any news increases future volatility
        \item Higher volatility $\rightarrow$ higher discount rate $\rightarrow$ lower prices
        \item Bad news: double effect (fundamentals + volatility feedback)
        \item Result: ``No news is good news''
    \end{itemize}
\end{enumerate}
\end{frame}

\begin{frame}[nologo]{Asymmetric Models: GJR-GARCH}
\textbf{GJR-GARCH(1,1)} ---  \href{https://doi.org/10.1111/j.1540-6261.1993.tb05128.x}{Glosten, Jagannathan, Runkle (1993)}:
$$\sigma_t^2 = \omega + \alpha\varepsilon_{t-1}^2 + \gamma\varepsilon_{t-1}^2\mathbb{I}_{\varepsilon_{t-1}<0} + \beta \sigma_{t-1}^2$$

where $\mathbb{I}_{\varepsilon_{t-1}<0} = 1$ if $\varepsilon_{t-1} < 0$, zero otherwise

\vspace{0.4cm}
\textbf{Impact of Shocks:}
\begin{itemize}
    \item \textbf{Positive shock} ($\varepsilon_{t-1} > 0$): impact = $\alpha$
    \item \textbf{Negative shock} ($\varepsilon_{t-1} < 0$): impact = $\alpha + \gamma$
\end{itemize}

\vspace{0.3cm}
\textbf{Interpretation:} If $\gamma > 0$, negative shocks increase volatility more

\textbf{Stationarity:} $\alpha + \beta + \gamma/2 < 1$ (for symmetric $z_t$)
\end{frame}

\begin{frame}[nologo]{Other Asymmetric Models: EGARCH \& APARCH}
\textbf{EGARCH(1,1)} --- \href{https://doi.org/10.2307/2938260}{Nelson (1991)}:
$$\log(\sigma_t^2) = \omega + \gamma_1 z_{t-1} + \alpha_1(|z_{t-1}| - E[|z_{t-1}|]) + \beta_1\log(\sigma_{t-1}^2)$$
\begin{itemize}
    \item Models $\log(\sigma_t^2)$ $\rightarrow$ no non-negativity constraints
    \item $\gamma_1 < 0$ implies leverage effect
\end{itemize}

\vspace{0.3cm}
\textbf{APARCH(1,1)} --- \href{https://www.sciencedirect.com/science/article/pii/092753989390006D}{Ding, Granger, Engle (1993)}:
$$(\sigma_t^2)^{\delta/2} = \omega + \alpha(|\varepsilon_{t-1}| - \gamma\varepsilon_{t-1})^\delta + \beta (\sigma_{t-1}^2)^{\delta/2}$$
\begin{itemize}
    \item $\delta > 0$: power parameter (data-driven)
    \item $\gamma \in (-1,1)$: asymmetry parameter
    \item Special case: $\delta = 2$, $\gamma = 0$ gives standard GARCH
\end{itemize}

\vspace{0.3cm}
\textbf{Note:} While flexible, GJR-GARCH often preferred for simplicity and interpretability
\end{frame}

\begin{frame}[nologo]{News Impact Curves}
\textbf{Definition:} How does today's shock $\varepsilon_{t-1}$ affect tomorrow's volatility $\sigma_t^2$?

\begin{center}
\includegraphics[width=0.85\textwidth]{assets/other/news_impact_curve.pdf}
\end{center}

\end{frame}

\begin{frame}[nologo]{HAR Model: Heterogeneous AutoRegressive}
\textbf{HAR Model} --- \href{https://academic.oup.com/jfec/article-abstract/7/2/174/856522}{Corsi (2009)}:
$$\text{RV}_{t,t+h} = \beta_0 + \beta_D \text{RV}_{t-1,t} + \beta_W \text{RV}_{t-5,t} + \beta_M \text{RV}_{t-22,t} + \varepsilon_t$$

\textbf{Key Features:}
\begin{itemize}
    \item Uses \textbf{realized volatility (RV)} from high-frequency data
    \item Aggregates volatility over multiple horizons:
    \begin{itemize}
        \item Daily: $\text{RV}_{t-1,t}$
        \item Weekly: $\text{RV}_{t-5,t} = \frac{1}{5}\sum_{i=1}^{5}\text{RV}_{t-i,t-i+1}$
        \item Monthly: $\text{RV}_{t-22,t} = \frac{1}{22}\sum_{i=1}^{22}\text{RV}_{t-i,t-i+1}$
    \end{itemize}
    \item Captures heterogeneous market participants (short/long horizons)
\end{itemize}

\textbf{Advantages:}
\begin{itemize}
    \item Simple linear regression (fast estimation)
    \item Captures long memory without fractional differencing
    \item Often outperforms GARCH for realized volatility forecasting
\end{itemize}
\end{frame}

\begin{frame}[nologo]{HAR-X: Adding External Information}
\begin{block}{HAR-X Model}
$$\text{RV}_{t+1} = \beta_0 + \beta_D \text{RV}_t + \beta_W \text{RV}_{t-5,t} + \beta_M \text{RV}_{t-22,t} + \gamma X_t + \varepsilon_{t+1}$$
\end{block}

\textbf{Common Extensions:}
\begin{itemize}
    \item \textbf{HAR-RV-CJ:} Separate continuous ($C_t$) and jump ($J_t$) components
    \item \textbf{HAR-RSV:} Realized semivariance (leverage effect)
    \item \textbf{HAR-VIX:} Include implied volatility as predictor
\end{itemize}

\vspace{0.2cm}
\begin{block}{Good to know:}
\begin{itemize}
    \item Cannot use future information (look-ahead bias)
    \item Linear structure allows ``easy'' extension with Machine Learning (\href{https://academic.oup.com/jfec/article-abstract/21/5/1680/6612759}{Christensen, Siggaard, \& Veliyev, 2023})
\end{itemize}
\end{block}
\end{frame}


\begin{frame}[nologo]{Why So Many Models?}
\textbf{Each model addresses specific stylized facts:}

\begin{itemize}
    \item \textbf{Asymmetry}: GJR-GARCH, EGARCH, APARCH, TGARCH
    \item \textbf{Long Memory}: FIGARCH, HYGARCH, FIEGARCH
    \item \textbf{Structural Breaks}: Component GARCH, MRS-GARCH
    \item \textbf{Multivariate}: DCC, BEKK, OGARCH
    \item \textbf{Realized Volatility}: HAR, MIDAS-GARCH-RV
\end{itemize}

\vspace{0.3cm}
\textbf{Trade-offs:}
\begin{itemize}
    \item \textcolor{blue}{More complexity} $\rightarrow$ better fit to stylized facts
    \item \textcolor{red}{More parameters} $\rightarrow$ overfitting risk, harder estimation
    \item \textcolor{orange}{Simplicity} often wins in out-of-sample forecasting
\end{itemize}

\vspace{0.3cm}
\textbf{Practical Advice:} Start with GARCH(1,1), add complexity only if needed!
\end{frame}

% =====================================================
% SECTION 3: FORECASTING VOLATILITY
% =====================================================
\section{Forecasting Volatility}

\begin{frame}[nologo]{Why Forecast Volatility?}
\textbf{Key Applications:}
\begin{itemize}
    \item \textbf{Risk Management}: VaR, ES, portfolio risk
    \item \textbf{Regulatory Compliance}: Basel III, Solvency II
    \item \textbf{Option Pricing}: Volatility is a key input
    \item \textbf{Asset Allocation}: Risk-adjusted position sizing
    \item \textbf{Trading Strategies}: Volatility targeting, mean reversion
\end{itemize}

\vspace{0.3cm}
\textbf{Forecast Horizons:}
\begin{itemize}
    \item \textbf{Next day}: Intraday trading, daily risk reports
    \item \textbf{Next week}: Weekly rebalancing, tactical trading
    \item \textbf{Longer term}: Strategic allocation, option pricing
\end{itemize}
\end{frame}

\begin{frame}[nologo]{One-Step Ahead Forecast (Next Day)}
\textbf{GARCH(1,1) one-step ahead:}

\begin{block}{Next Day Volatility}
$$\hat{\sigma}_{t+1}^2 = \omega + \alpha r_t^2 + \beta \sigma_t^2$$
\end{block}

\textbf{Procedure:}
\begin{enumerate}
    \item Estimate GARCH parameters using historical data
    \item Compute current volatility $\sigma_t^2$
    \item Observe current return $r_t$
    \item Plug into formula to get $\hat{\sigma}_{t+1}^2$
\end{enumerate}

\end{frame}

\begin{frame}[nologo]{Multi-Step Ahead Forecasts}
\textbf{Challenge:} How to forecast $\hat{\sigma}_{t+h}^2$ for $h > 1$?

\textbf{Key Insight:} Future returns are unpredictable (EMH), so $E[r_{t+j}^2] = \sigma_{t+j}^2$

\begin{block}{GARCH(1,1) Multi-Step Forecast}
$$\hat{\sigma}_{t+h}^2 = \sigma_L^2 + (\alpha + \beta)^{h-1}(\hat{\sigma}_{t+1}^2 - \sigma_L^2)$$
\end{block}

where $\sigma_L^2 = \frac{\omega}{1-\alpha-\beta}$ is the long-run variance

\textbf{Observations:}
\begin{itemize}
    \item Exponential decay toward long-run variance
    \item Rate of decay: $\alpha + \beta$ (persistence parameter)
    \item As $h \rightarrow \infty$: $\hat{\sigma}_{t+h}^2 \rightarrow \sigma_L^2$
\end{itemize}
\end{frame}

\begin{frame}[nologo]{Example: Weekly Forecast}
\textbf{Question:} Forecast volatility for next week (5 trading days)

\textbf{Approach 1: Direct Multi-Step}
$$\hat{\sigma}_{t+5}^2 = \sigma_L^2 + (\alpha + \beta)^{4}(\hat{\sigma}_{t+1}^2 - \sigma_L^2)$$

\textbf{Approach 2: Path-Based (Monte Carlo)}
\begin{enumerate}
    \item Simulate many return paths
    \item For each path, compute daily GARCH updates
    \item Average volatility at day $t+5$
\end{enumerate}

\textbf{Approach 3: Aggregation}
$$\sigma_{t:t+5}^2 = \sum_{i=1}^{5} \hat{\sigma}_{t+i}^2$$

This gives the 5-day variance (for 5-day VaR)
\end{frame}

\begin{frame}[nologo]{Direct vs. Iterative Multi-Step Forecasts}
\textbf{\href{https://www.annualreviews.org/content/journals/10.1146/annurev-financial-110217-022808}{Ghysels, Plazzi, Valkanov, Rubia, \& Dossani (2019)}:} Direct Versus Iterated Multiperiod Volatility Forecasts

\vspace{0.2cm}
\textbf{Two Approaches for Multi-Step Ahead Forecasts:}

\begin{columns}[T]
\column{0.48\textwidth}
\textbf{1. Iterative (Plug-in):}
\begin{itemize}
    \item Forecast 1-step: $\hat{\sigma}_{t+1|t}^2$
    \item Use as input: $\hat{\sigma}_{t+2|t}^2$
    \item Iterate to horizon $h$
    \item Standard GARCH approach
\end{itemize}
\textcolor{blue}{$\checkmark$ Efficient if model correct}\\
\textcolor{red}{$\times$ Errors compound over horizons}

\column{0.48\textwidth}
\textbf{2. Direct Method:}
\begin{itemize}
    \item Estimate separate model for $h$
    \item $\sigma_{t+h}^2 = f(X_t) + \varepsilon_{t+h}$
    \item No intermediate forecasts
    \item Tailored to each horizon
\end{itemize}
\textcolor{blue}{$\checkmark$ Robust to misspecification}\\
\textcolor{red}{$\times$ Less efficient; many models}
\end{columns}

\vspace{0.3cm}
\textbf{Key Finding:} No universally dominant method---depends on horizon and model
\end{frame}


\begin{frame}[nologo]{Improving Misspecified Models}

\textbf{\href{https://www.sciencedirect.com/science/article/pii/S0304407624001131}{Oh \& Patton (2024)}:} Better the devil you know: Improved forecasts from imperfect models

\textbf{Key Idea:} Even good models are inevitably misspecified

\begin{itemize}
    \item Improve forecasts from a \textit{known} misspecified model
    \item Use \textbf{local M estimation} (local OLS, local MLE)
    \item Weight observations using a state variable correlated with misspecification
\end{itemize}

\vspace{0.3cm}
\textbf{Method:}
\begin{itemize}
    \item Upweight past data that looks similar to current environment
    \item Downweight dissimilar observations
    \item Examples: use realized volatility, VIX, or time as state variable
\end{itemize}

\vspace{0.3cm}
\textbf{Applications:}
\begin{itemize}
    \item GARCH and HAR volatility forecasting
    \item VaR and ES forecasting
    \item Yield curve forecasting
    \item \textit{Significant forecast improvements across all applications}
\end{itemize}
\end{frame}

\begin{frame}[nologo]{Forecast Evaluation: Loss Functions}
\textbf{How do we know if our forecasts are good?}

\textbf{Challenge:} True volatility $\sigma_t^2$ is unobservable!

\textbf{Proxies:} Squared returns $r_t^2$, realized volatility (from high-frequency data)

\vspace{0.3cm}
\textbf{Common Loss Functions:}
\begin{itemize}
    \item \textbf{MSE}: Mean Squared Error
    $$\text{MSE} = \frac{1}{T}\sum_{t=1}^{T}(\hat{\sigma}_t^2 - \sigma_t^{2,\text{realized}})^2$$

    \item \textbf{QLIKE}: Quasi-likelihood loss (robust, preferred)
    $$\text{QLIKE} = \frac{1}{T}\sum_{t=1}^{T}\left(\frac{\sigma_t^{2,\text{realized}}}{\hat{\sigma}_t^2} - \log\frac{\sigma_t^{2,\text{realized}}}{\hat{\sigma}_t^2} - 1\right)$$
\end{itemize}

\textbf{Why QLIKE?} Less sensitive to outliers than MSE (\href{https://www.sciencedirect.com/science/article/pii/S030440761000076X}{Patton, 2011})
\end{frame}

\begin{frame}[nologo]{Model Comparison}

\textbf{\href{https://onlinelibrary.wiley.com/doi/full/10.1002/jae.800}{Hansen \& Lunde (2005)}:} A forecast comparison of volatility models: Does anything beat a GARCH(1,1)?


\begin{itemize}
    \item Compared \textbf{330 ARCH-type models} on exchange rate data
    \item Used realized volatility from high-frequency data as proxy
    \item Evaluated with multiple loss functions
    \item Superior Predictive Ability test
\end{itemize}

\vspace{0.3cm}
\begin{block}{Key Finding}
``We find no evidence that any model outperforms GARCH(1,1)''
\end{block}

\vspace{0.3cm}
\textbf{Implications:}
\begin{itemize}
    \item Parsimony matters---complex models don't always win
    \item GARCH(1,1) is a robust benchmark
    \item Results may differ for other asset classes (equities, commodities)
\end{itemize}
\end{frame}

\begin{frame}[nologo]{Model Confidence Set: \href{https://onlinelibrary.wiley.com/doi/abs/10.3982/ECTA5771}{Hansen (2011)}}
\textbf{Problem:} Comparing many models requires statistical rigor

\textbf{\href{https://onlinelibrary.wiley.com/doi/abs/10.3982/ECTA5771}{Hansen (2011)}:} Model Confidence Set (MCS)
\begin{itemize}
    \item Statistical framework for comparing multiple forecasts
    \item Accounts for model uncertainty and multiple testing
    \item Identifies set of ``superior'' models at given confidence level
\end{itemize}

\vspace{0.3cm}
\textbf{MCS Procedure:}
\begin{enumerate}
    \item Compute pairwise loss differentials: $d_{ij,t} = L_{i,t} - L_{j,t}$
    \item Test $H_0$: All models have equal expected loss
    \item If rejected, eliminate worst model; repeat until $H_0$ not rejected
    \item Result: Set $\widehat{\mathcal{M}}^*_{1-\alpha}$ of statistically indistinguishable models
\end{enumerate}

\vspace{0.3cm}
\textbf{Test Statistics:}
\begin{itemize}
    \item \textbf{Range statistic} $T_R$: Based on $\max_{i,j} |\bar{d}_{ij}|$
    \item \textbf{Semi-quadratic} $T_{SQ}$: Based on $\sum_{i<j} \bar{d}_{ij}^2$
    \item Bootstrap for p-values (block bootstrap for dependence)
\end{itemize}
\end{frame}

\begin{frame}[nologo]{Arbitrary Modeling Choices}
\textbf{Problem:} Many implementation choices are not dictated by theory!

\begin{columns}[T]
\column{0.48\textwidth}
\begin{itemize}
    \item \textbf{Training Window}: Rolling (250/500/1000 days) vs.\ expanding?
    \item \textbf{Forecasting Method}: Iterative vs.\ direct vs.\ scaling?
    \item \textbf{Model Selection}: GARCH(1,1) vs.\ GJR-GARCH? Normal vs.\ Student-t?
    \item \textbf{Data Frequency}: Daily vs.\ weekly? High-frequency intervals?
\end{itemize}

\column{0.48\textwidth}
\begin{itemize}
    \item \textbf{Return Calculation}: Log returns required; percentage or decimal?
    \item \textbf{Rebalancing}: Daily/weekly/monthly? Transaction costs matter
    \item \textbf{Evaluation Period}: Results vary across subperiods
    \item \textbf{Evaluation Method}: RMSE, QLIKE, R\textsuperscript{2},...
\end{itemize}
\end{columns}

\end{frame}

\begin{frame}[nologo]{Implications of Arbitrary Choices}
\textbf{Key Insights:}

\begin{itemize}
    \item \textbf{No ``correct'' answer}: Theory often silent on these choices
    \item \textbf{Sensitivity analysis crucial}: Test multiple specifications
    \item \textbf{Robustness matters}: Prefer results that hold across choices
    \item \textbf{Context-dependent}: Optimal choices vary by application
\end{itemize}

\vspace{0.3cm}
\textbf{Best Practices:}

\begin{enumerate}
    \item \textbf{Report multiple specifications}: Don't cherry-pick
    \item \textbf{Use industry standards when available}: e.g., 250-day window
    \item \textbf{Economic intuition}: Choose based on application needs
    \item \textbf{Out-of-sample validation}: Test on holdout data
    \item \textbf{Be transparent}: Document all choices clearly
\end{enumerate}

\vspace{-0.7cm}
\begin{center}
\begin{minipage}{0.60\textwidth}
\vspace{0.3cm}
\raggedleft
\textit{``With great flexibility comes\\great responsibility''}\\[0.1cm]
{\small\mbox{- Uncle Ben (and researchers choosing model specifications)}}
\end{minipage}%
\hspace{0.6cm}%
\begin{minipage}{0.2\textwidth}
\vspace{-0.6cm}
\IfFileExists{assets/other/spiderman.png}{%
    \includegraphics[width=\textwidth]{assets/other/spiderman.png}
}{%
    \IfFileExists{assets/other/spiderman.jpg}{%
        \includegraphics[width=\textwidth]{assets/other/spiderman.jpg}
    }{}%
}
\end{minipage}
\end{center}
\end{frame}

\begin{frame}[nologo]{Real-Time Forecasting: The Rear-View Mirror Problem}
\textbf{\href{https://doi.org/10.1093/rfs/hhx098}{Ghysels, Horan, \& Moench (2018)}:} Forecasting through the Rearview Mirror: Data Revisions and Bond Return Predictability

\vspace{0.2cm}
\textbf{Challenge:} Data revisions affect real-time forecasts

\begin{block}{The Problem}
\begin{itemize}
    \item Realized volatility estimated from high-frequency data
    \item Initial estimates revised as more data arrives
    \item Models estimated on ``final'' data perform poorly in real-time
\end{itemize}
\end{block}

\textbf{Real-Time vs.\ Revised Data:}
\begin{itemize}
    \item \textbf{Ex-post (revised):} Clean, complete data with revisions
    \item \textbf{Real-time:} What you actually have when forecasting
    \item \textbf{Performance gap:} Models look better on revised data
\end{itemize}

\textbf{Solutions:}
\begin{itemize}
    \item Use real-time data vintages for estimation (e.g.\ \href{https://alfred.stlouisfed.org/}{ALFRED})
    \item Account for data uncertainty in forecasts
\end{itemize}
\end{frame}

% =====================================================
% SECTION 4: TRADING WITH VOLATILITY
% =====================================================
\section{Forward-Looking Volatility as a Trading Signal}

\begin{frame}[nologo]{From Risk Management to Trading}
\textbf{Insight:} Volatility forecasts contain valuable information for trading

\textbf{Trading Approaches:}
\begin{enumerate}
    \item \textbf{Volatility Targeting}: Adjust position size based on forecast
    \item \textbf{Mean Reversion Trading}: Trade volatility itself
    \item \textbf{Volatility Timing}: Shift between assets based on volatility
    \item \textbf{Options Trading}: Exploit differences between forecast and implied vol
\end{enumerate}
\end{frame}

\begin{frame}[nologo]{Strategy 1: Volatility Targeting}
\textbf{Concept:} Maintain constant risk exposure

\begin{block}{Position Sizing Formula}
$$w_t = \frac{\sigma_{\text{target}}}{\hat{\sigma}_{t+1}}$$
\end{block}

where:
\begin{itemize}
    \item $w_t$ = position weight at time $t$
    \item $\sigma_{\text{target}}$ = target portfolio volatility (e.g., 10\% annual)
    \item $\hat{\sigma}_{t+1}$ = forecasted volatility
\end{itemize}

\textbf{Result:}
\begin{itemize}
    \item Reduce exposure when volatility is high
    \item Increase exposure when volatility is low
    \item Smoother return profile
    \item Better risk-adjusted returns (higher Sharpe ratio)
\end{itemize}
\end{frame}

\begin{frame}[nologo]{Strategy 2: Volatility Mean Reversion}
\textbf{Stylized Fact:} Volatility mean-reverts

\textbf{Trading Logic:}
\begin{itemize}
    \item When $\hat{\sigma}_t > \sigma_L + k \cdot \text{std}(\sigma)$: short volatility
    \item When $\hat{\sigma}_t < \sigma_L - k \cdot \text{std}(\sigma)$: long volatility
\end{itemize}

\textbf{Implementation:}
\begin{enumerate}
    \item Options: Sell options when IV high, buy when IV low
    \item VIX futures: Short VIX futures in high vol regimes
    \item ETFs: Trade volatility ETFs (e.g., VXX, SVXY)
\end{enumerate}

\textbf{Warning:}
\begin{itemize}
    \item High volatility can persist longer than expected
    \item Risk of large losses in crisis periods
    \item Need careful risk management
\end{itemize}
\end{frame}

\begin{frame}[nologo]{Strategy 3: Regime Switching}
\textbf{Idea:} Different assets perform better in different volatility regimes

\begin{center}
\begin{tabular}{lcc}
\toprule
\textbf{Volatility Regime} & \textbf{Overweight} & \textbf{Underweight} \\
\midrule
Low ($\hat{\sigma}_t < \sigma_L$) & Equities & Bonds, Cash \\
Medium ($\hat{\sigma}_t \approx \sigma_L$) & Balanced & - \\
High ($\hat{\sigma}_t > \sigma_L$) & Bonds, Gold & Equities \\
\bottomrule
\end{tabular}
\end{center}

\textbf{Implementation:}
\begin{itemize}
    \item Define volatility thresholds (e.g., percentiles)
    \item Rebalance when regime changes
    \item Can combine with momentum/value signals
\end{itemize}
\end{frame}

\begin{frame}[nologo]{Strategy 4: Forecast vs. Implied Volatility}
\textbf{Two Volatility Measures:}
\begin{itemize}
    \item \textbf{Historical/GARCH}: Based on past returns
    \item \textbf{Implied Volatility (IV)}: From option prices
\end{itemize}

\textbf{Trading Signal:}
$$\text{Signal}_t = \hat{\sigma}_{t+1}^{\text{GARCH}} - \text{IV}_t$$

\textbf{Strategy:}
\begin{itemize}
    \item If Signal $> 0$: Buy options (IV underpriced)
    \item If Signal $< 0$: Sell options (IV overpriced)
\end{itemize}

\textbf{Refinements:}
\begin{itemize}
    \item Use term structure of implied volatility
    \item Adjust for volatility risk premium
    \item Consider transaction costs and bid-ask spreads
\end{itemize}
\end{frame}

\begin{frame}[nologo]{Practical Considerations}
\textbf{Before Trading on Volatility Forecasts:}

\begin{enumerate}
    \item \textbf{Backtest thoroughly}: Out-of-sample testing essential
    \item \textbf{Transaction costs}: Rebalancing costs can erode profits
    \item \textbf{Model risk}: GARCH may not always be accurate
    \item \textbf{Regime changes}: Parameters may shift over time
    \item \textbf{Tail risk}: Volatility strategies can have severe drawdowns
    \item \textbf{Leverage}: Many vol strategies use leverage (dangerous!)
\end{enumerate}

\vspace{0.3cm}
\textbf{Best Practices:}
\begin{itemize}
    \item Combine multiple models (ensemble approach)
    \item Use rolling window estimation
    \item Implement strict risk limits
    \item Monitor model performance continuously
\end{itemize}
\end{frame}

% =====================================================
% CONCLUSION
% =====================================================
\section{Conclusion}

\begin{frame}[nologo]{Summary}
\textbf{What We Covered:}
\begin{itemize}
    \item \textbf{Foundation}: EWMA, GARCH, MLE, VaR
    \item \textbf{Stylized Facts}: Clustering, mean reversion, leverage effect
    \item \textbf{Forecasting}: One-step and multi-step ahead predictions
    \item \textbf{Trading}: Using volatility forecasts as trading signals
\end{itemize}

\vspace{0.3cm}
\textbf{Key Takeaways:}
\begin{enumerate}
    \item Volatility is predictable (unlike returns!)
    \item GARCH models capture key empirical features
    \item Forecasts enable better risk management and trading
    \item Always backtest and validate your models
\end{enumerate}
\end{frame}

{
\setbeamertemplate{footline}{}
\begin{frame}[plain]
  \begin{tikzpicture}[remember picture, overlay]
    \fill[darkbg] (current page.south west) rectangle (current page.north east);
    \node[anchor=center, opacity=0.25] at (current page.center) {
      \includegraphics[width=\paperwidth, height=\paperheight]{\uubg{endpage_light_green.png}}
    };
    \node[anchor=center, align=center, text=uuwhite] at (current page.center) {
      {\Huge Thank You!}\\[0.5cm]
      {\Large Questions?}\\[0.5cm]
      {\normalsize }
    };
    \node[anchor=south west] at ([xshift=0.8cm, yshift=0.8cm]current page.south west) {
      \includegraphics[height=2cm]{\uulogo{UU_logo_2021_EN_WHITE.png}}
    };
  \end{tikzpicture}
\end{frame}
}


% =====================================================
% APPENDIX
% =====================================================
\appendix

\section{Appendix}

\begin{frame}[nologo]{Appendix Overview}
\textbf{Additional topics and resources:}

\vspace{0.2cm}
\begin{columns}[T]
\column{0.48\textwidth}
\begin{itemize}
    \item \textbf{A:} Long Memory in Volatility
    \begin{itemize}
        \item Fractional integration, FIGARCH
    \end{itemize}

    \vspace{0.15cm}
    \item \textbf{B:} Structural Breaks \& Regime Switching
    \begin{itemize}
        \item Markov-Switching models
    \end{itemize}

    \vspace{0.15cm}
    \item \textbf{C:} The GARCH Model Zoo
    \begin{itemize}
        \item Overview of model variants
    \end{itemize}
\end{itemize}

\column{0.48\textwidth}
\begin{itemize}
    \item \textbf{D:} Software Resources
    \begin{itemize}
        \item Python packages
    \end{itemize}

    \vspace{0.15cm}
    \item \textbf{E:} Workshop Preparation
    \begin{itemize}
        \item Data access (WRDS, Yahoo Finance)
    \end{itemize}
\end{itemize}
\end{columns}
\end{frame}


% =====================================================
% APPENDIX A: LONG MEMORY
% =====================================================
\subsection{Appendix A: Long Memory in Volatility}

\begin{frame}[nologo]{Stylized Fact: Long Memory}
\textbf{Observation:} Autocorrelations of squared returns decay very slowly

\begin{center}
\includegraphics[width=0.95\textwidth]{assets/other/acf_long_memory.pdf}
\end{center}

\vspace{-0.1cm}
\textbf{Implication:} Volatility shocks have long-lasting effects

\textbf{Problem:} Standard GARCH implies exponential decay
\end{frame}

\begin{frame}[nologo]{Fractional Integration}
\textbf{Motivation:} Model hyperbolic decay in ACF

\textbf{Fractional Differencing Operator:}
$$(1-L)^d$$

where $L$ is the lag operator and $0 < d < 1$

\vspace{0.3cm}
\textbf{Properties:}
\begin{itemize}
    \item $d = 0$: No differencing (standard GARCH)
    \item $0 < d < 1$: Long memory (hyperbolic decay)
    \item $d = 1$: Integrated (IGARCH, non-stationary)
\end{itemize}

\textbf{Key Innovation:} One parameter $d$ captures long memory!
\end{frame}

\begin{frame}[nologo]{FIGARCH Model}
\begin{block}{FIGARCH --- \href{https://www.sciencedirect.com/science/article/pii/S0304407695017496}{Baillie, Bollerslev, Mikkelsen (1996)}}
$$\sigma_t^2 = \frac{\omega}{1-\beta_1} + \left[1 - \frac{(1-\phi_1 L)(1-L)^d}{1-\beta_1 L}\right]\varepsilon_t^2$$
\end{block}

\textbf{Can also be written as:}
$$\sigma_t^2 = \frac{\omega}{1-\beta_1} + \sum_{i=1}^{\infty}\lambda_i^{FI}\varepsilon_{t-i}^2$$

\textbf{Parameters:}
\begin{itemize}
    \item $d$: fractional differencing (long memory parameter)
    \item $\phi_1$: AR parameter (short-run dynamics)
    \item $\beta_1$: GARCH persistence
\end{itemize}

\textbf{Challenge:} Estimation is slow (infinite summation)

\textbf{Solution:} Fast Fourier Transform (Klein \& Walther, 2017)
\end{frame}

\begin{frame}[nologo]{Other Long Memory Models}
\textbf{Extensions combining long memory with asymmetry:}

\begin{itemize}
    \item \textbf{Hyperbolic GARCH (HYGARCH)} --- \href{https://doi.org/10.1198/073500103288619359}{Davidson (2004)}
    \begin{itemize}
        \item Partial fractional integration
        \item Interpolates between GARCH and FIGARCH
    \end{itemize}

    \item \textbf{FIEGARCH} --- \href{https://www.sciencedirect.com/science/article/pii/0304407695017364}{Bollerslev \& Mikkelsen (1996)}
    \begin{itemize}
        \item Long memory + asymmetry (EGARCH structure)
        \item Models $\log(\sigma_t^2)$ with fractional differencing
    \end{itemize}

    \item \textbf{FIAPARCH} --- \href{https://doi.org/10.1002/(SICI)1099-1255(199801/02)13:1<49::AID-JAE459>3.0.CO;2-O}{Tse (1998)}
    \begin{itemize}
        \item Long memory + asymmetry + power transformation
        \item Most flexible, but many parameters
    \end{itemize}
\end{itemize}

\textbf{Trade-off:} Flexibility vs. computational cost vs. overfitting risk
\end{frame}

% =====================================================
% APPENDIX B: STRUCTURAL BREAKS
% =====================================================
\subsection{Appendix B: Structural Breaks and Regime Switching}

\begin{frame}[nologo]{Stylized Fact: Structural Breaks}
\textbf{Problem:} Standard GARCH assumes fixed parameters over time

\textbf{Reality:} Volatility dynamics shift due to:
\begin{itemize}
    \item Economic cycles (recession vs. expansion)
    \item Policy changes (e.g., central bank interventions)
    \item Market crises or extreme events
    \item Regulatory or technological shifts
\end{itemize}

\vspace{0.3cm}
\textbf{Consequence:} Parameter bias and spurious persistence

\vspace{0.3cm}
\begin{center}
\textit{``The strong persistence in variance is due to structural changes''}\\
--- \href{https://www.tandfonline.com/doi/abs/10.1080/07350015.1994.10524546}{Cai (1994)}
\end{center}

\textbf{Solution:} Markov-Regime-Switching (MRS) models
\end{frame}

\begin{frame}[nologo]{Markov-Regime-Switching Framework}
\textbf{Idea:} At each time $t$, the process is in one of $R$ unobserved states

\textbf{State Variable:} $S_t \in \{1, 2, \ldots, R\}$

\textbf{Transition Probabilities:}
$$P_{i,j} = P[S_t = j | S_{t-1} = i]$$

\begin{block}{Transition Matrix (for 2 regimes)}
$$\mathbf{P} = \begin{bmatrix}
P_{1,1} & P_{1,2}\\
P_{2,1} & P_{2,2}
\end{bmatrix}$$
\end{block}

\textbf{Properties:}
\begin{itemize}
    \item Rows sum to 1: $\sum_j P_{i,j} = 1$
    \item Diagonal elements measure regime persistence
    \item Framework by \href{https://doi.org/10.2307/1912559}{Hamilton (1989)}
\end{itemize}
\end{frame}

\begin{frame}[nologo]{MRS-ARCH Model}
\begin{block}{MRS-ARCH --- \href{https://www.sciencedirect.com/science/article/pii/0304407694900671}{Hamilton \& Susmel (1994)}, \href{https://www.tandfonline.com/doi/abs/10.1080/07350015.1994.10524546}{Cai (1994)}}
\begin{align*}
r_t &= \mu_{t,S_t} + \sigma_{t,S_t}\cdot z_t\\
\sigma_{t,S_t}^2 &= \omega_{S_t} + \sum_{i=1}^{q}\alpha_{i,S_t}\varepsilon_{t-i}^2
\end{align*}
\end{block}

\textbf{Key Feature:} All parameters depend on current regime $S_t$

\textbf{Example (2 regimes):}
\begin{itemize}
    \item Regime 1: ``Low volatility'' (normal times)
    \item Regime 2: ``High volatility'' (crisis periods)
\end{itemize}

\textbf{Advantages:}
\begin{itemize}
    \item Captures time-varying volatility dynamics
    \item Identifies distinct market conditions
    \item Reduces spurious persistence
\end{itemize}
\end{frame}

\begin{frame}[nologo]{MRS-GARCH: Path-Dependency Problem}
\textbf{Na\"ive MRS-GARCH:}
$$\sigma_{t,S_t}^2 = \omega_{S_t} + \alpha_{S_t}\varepsilon_{t-1}^2 + \beta_{S_t}\sigma_{t-1}^2$$

\textbf{Problem:} What is $\sigma_{t-1}^2$?
\begin{itemize}
    \item Depends on $S_{t-1}$, which could be any of $R$ states
    \item To compute $\sigma_{t,S_t}^2$, need $\sigma_{t-1,S_{t-1}}^2$ for all possible $S_{t-1}$
    \item This creates $R^t$ paths (exponentially growing!)
\end{itemize}

\textbf{Solutions:}
\begin{enumerate}
    \item \textbf{\href{https://doi.org/10.1016/0304-405X(96)00875-6}{Gray (1996)}:} Use $E[\sigma_{t-1}^2|\mathcal{F}_{t-2}]$
    \item \textbf{\href{https://link.springer.com/article/10.1007/s001810100100}{Klaassen (2002)}:} Use $E[\sigma_{t-1}^2|\mathcal{F}_{t-1}]$ (more accurate)
    \item \textbf{\href{https://academic.oup.com/jfec/article-abstract/2/4/493/900480}{Haas et al. (2004)}:} Separate variance paths per regime
\end{enumerate}
\end{frame}


% =====================================================
% APPENDIX C: GARCH ZOO
% =====================================================
\subsection{Appendix C: The GARCH Model Zoo}

\begin{frame}[nologo]{The GARCH Zoo/Soup/Alphabet}
\begin{table}
\centering
\begin{tabular}{ll|ll}
\toprule
A & APARCH & N & NGARCH\\
B & Beta-GARCH & O & OGARCH \\
C & CGARCH & P & PGARCH\\
D & DCC & Q & QGARCH\\
E & EGARCH & R & RGARCH\\
F & FIGARCH & S & SGARCH\\
G & GJR-GARCH & T & TGARCH\\
H & HYGARCH & U & UGARCH\\
I & IGARCH & V & VGARCH\\
J & Jump-GARCH & W & Weak GARCH\\
K & KGARCH & X & \\
L & LMGARCH & Y & \\
M & MGARCH & Z & ZARCH\\
\bottomrule
\end{tabular}
\caption{The GARCH zoo/soup/alphabet. The list is not exhaustive. See \href{https://papers.ssrn.com/sol3/papers.cfm?abstract_id=1263250}{ Bollerslev (2008)} for a wider overview.}
\end{table}

\vspace{0.3cm}
\textbf{Key Insight:} Despite the variety, GARCH(1,1) often performs remarkably well!\\
\textit{(Hansen \& Lunde, 2005: ``Hard to beat GARCH(1,1) for exchange rates'')}
\end{frame}

% =====================================================
% APPENDIX D: SOFTWARE RESOURCES
% =====================================================
\subsection{Appendix D: Software Resources}

\begin{frame}[nologo]{Software Resources: Python Implementation}
\textbf{Core Volatility Modeling:}
\begin{itemize}
    \item \texttt{arch}: Kevin Sheppard's ARCH/GARCH library
    \begin{itemize}
        \item GARCH, EGARCH, GJR-GARCH, FIGARCH
        \item Multiple distributions (Normal, Student-t, skewed-t, GED)
        \item Comprehensive diagnostic tools
    \end{itemize}
    \item \texttt{statsmodels}: Additional time series models and tests
\end{itemize}

\vspace{0.3cm}
\textbf{Data Acquisition:}
\begin{itemize}
    \item \texttt{wrds}: WRDS database access (Compustat)
    \item \texttt{yfinance}: Yahoo Finance data (free alternative)
    \item \texttt{pandas\_datareader}: Multiple data sources
\end{itemize}

\vspace{0.3cm}
\textbf{Supporting Tools:}
\begin{itemize}
    \item \texttt{pandas}: Data manipulation and time series
    \item \texttt{numpy}: Numerical computing
    \item \texttt{matplotlib/seaborn}: Visualization
    \item \texttt{scipy}: Optimization and statistical functions
\end{itemize}
\end{frame}

% =====================================================
% APPENDIX E: WORKSHOP PREPARATION
% =====================================================
\subsection{Appendix E: Workshop Preparation}

\begin{frame}[nologo]{Getting Started with Wharton Research Data Services (WRDS)}

\textbf{What is WRDS?}
\begin{itemize}
    \item Academic financial database (e.g., Compustat)
    \item Free access for university students/faculty
    \item Industry-standard data used in research
\end{itemize}

\vspace{0.3cm}
\textbf{Registration Steps:}
\begin{enumerate}
    \item Go to \texttt{\href{https://wrds-www.wharton.upenn.edu/}{wrds-www.wharton.upenn.edu}}
    \item Click ``Register for WRDS Account''
    \item Use your \textbf{university email address}
    \item Wait for approval (usually 1-2 business days)
    \item Check email for username confirmation
\end{enumerate}

\vspace{0.3cm}
\textbf{Important:} Register \textit{before} the workshop if you haven't already!
\end{frame}

\begin{frame}[nologo,fragile]{First-Time Setup in Jupyter Notebook}
\textbf{Step-by-step connection process:}

\begin{enumerate}
    \item \textbf{Import and connect:}
\begin{codebox}
  \color{uuwhite}
  \small
  \texttt{%
  \textcolor{uuyellow}{import} wrds\\
  db = wrds.Connection()
  }
\end{codebox}
    \item \textbf{Enter your WRDS username when prompted}
    \item \textbf{Enter your password} (password is hidden and won't show)
    \item \textbf{Save credentials?} (say yes)
\begin{verbatim}
Create a .pgpass file to store your password (y/n): y
\end{verbatim}
\item \textbf{After first setup:} (automatic login)
\begin{codebox}
  \color{uuwhite}
  \small
  \texttt{%
  db = wrds.Connection(wrds\_username='your\_username')
  }
\end{codebox}
\end{enumerate}



\end{frame}

\begin{frame}[nologo,fragile]{Alternative: Yahoo Finance (No Registration)}
\textbf{If WRDS doesn't work:} Use \texttt{yfinance} package

\begin{codebox}
  \color{uuwhite}
  \small
  \texttt{%
  \textcolor{uuyellow}{import} yfinance \textcolor{uuyellow}{as} yf\\
  \\
  \textcolor{uugreen}{\# Download S\&P 500 data}\\
  sp500 = yf.download('{\^{}}GSPC', start='2019-01-01',\\
  \hspace{2.2cm}end='2024-01-01')\\
  returns = sp500['Adj Close'].pct\_change()
  }
\end{codebox}


\textbf{Trade-offs:}\\
    \textcolor{green}{\checkmark} No registration needed, works immediately\\
    \textcolor{red}{$\times$} Less comprehensive than WRDS\\
    \textcolor{red}{$\times$} Sometimes rate-limited\\

\end{frame}



\end{document}
