% \DocumentMetadata{} % disabled for compatibility (avoids expl3 \ERROR at \begin{document})
\documentclass[aspectratio=169]{beamer}

% Scalable fonts (avoids "Font shape ... not available" at large sizes)
\usepackage[T1]{fontenc}
\usepackage{lmodern}
\renewcommand{\sfdefault}{lmss}

\usepackage{uu-aif-beamer}



% TikZ styles (defined in the preamble to avoid # tokens inside Beamer frames)
\tikzset{
  smallicon/.style={circle, draw=#1, line width=1pt, minimum size=1cm, fill=#1!10},
  topicbox/.style={anchor=west, align=left}
}

% Some Beamer setups can end up using LaTeX's list labels (\labelitemi, ...)
% without defining them; provide safe defaults.
\providecommand{\labelitemi}{\textbullet}
\providecommand{\labelitemii}{\textendash}
\providecommand{\labelitemiii}{\textasteriskcentered}
\providecommand{\labelitemiv}{\textperiodcentered}


% lecture_template.tex - Ready-to-use starter template%
% INSTRUCTIONS:
% 1. Copy this file to your lecture folder
% 2. Update the title slide parameters
% 3. Add your content between the section dividers
% 4. Compile with: pdflatex lecture_template.tex
%
% NOTE: This template assumes uu-aif-beamer is in a sibling directory.
% Adjust the asset paths if your folder structure differs.



\begin{document}

% =============================================================================
% TITLE SLIDE
% Update these three parameters for your lecture
% =============================================================================
\titleslide{Algorithms in Finance}{Introduction}{February 2026}

% =============================================================================
% SECTION 1
% =============================================================================
\sectiondivider{yellow_green.png}{Welcome}

% Add your content slides here
\begin{frame}{Welcome to Algorithms in Finance}
  \centering
  \textbf{\Large Powered by collaboration}
  \vspace{0.5cm}

  \begin{columns}[c]
    \column{0.5\textwidth}
    \centering
    \includegraphics[height=1.5cm]{\uuaifassets/logos/UU_logo_2021_EN_RGB.png}
    \vspace{0.5cm}
    
    \includegraphics[height=1.0cm]{\uuaifassets/logos/faculty-of-science-banner.png}

    \column{0.5\textwidth}
    \centering
    \includegraphics[height=1.5cm]{\uuaifassets/logos/optiver_logo.jpg}
    \vspace{0.5cm}
    
    \includegraphics[height=1.0cm]{\uuaifassets/logos/USE-banner.png}
  \end{columns}
\end{frame}

\begin{frame}{The Team}
  \vspace{0.2cm}
  \centering
  \begin{columns}[T]
    \column{0.33\textwidth}
    \centering
    \begin{tikzpicture}
      \node[anchor=south, inner sep=0] (image) at (0,0) {\includegraphics[height=5cm]{\uuaifassets/pictures/Thomas.jpg}};
      \node[anchor=south, yshift=0.05cm] at (image.south) {\textbf{Thomas Walther}};
    \end{tikzpicture}
    
    \column{0.33\textwidth}
    \centering
    \begin{tikzpicture}
      \node[anchor=south, inner sep=0] (image) at (0,0) {\includegraphics[height=5cm]{\uuaifassets/pictures/Robbert.jpg}};
      \node[anchor=south, yshift=0.05cm] (name) at (image.south) {\textbf{Robbert Pullen}};
      \node[anchor=north, yshift=-0.05cm] at (name.south) {\scriptsize \textbf{(Optiver)}};
    \end{tikzpicture}
    
    \column{0.33\textwidth}
    \centering
    \begin{tikzpicture}
      \node[anchor=south, inner sep=0] (image) at (0,0) {\includegraphics[height=5cm]{\uuaifassets/pictures/Mads.jpg}};
      \node[anchor=south, yshift=0.05cm] at (image.south) {\textbf{Mads Nielsen}};
    \end{tikzpicture}
  \end{columns}
\end{frame}

% Example of using the circicons
\begin{frame}{Course objectives}

  \vspace{0.5cm}

  \begin{columns}[T]
    \column{0.33\textwidth}
    \centering
    \circicon{uugreen}{\faChartLine}

    \vspace{0.3cm}
    \textbf{Understand pricing}

    \vspace{0.1cm}
    {\small as it \emph{actually} happens in financial markets.}

    \column{0.33\textwidth}
    \centering
    \circicon{uuyellow}{\faCode}

    \vspace{0.3cm}
    \textbf{Implement algorithms}

    \vspace{0.1cm}
    {\small for trading.}

    \column{0.33\textwidth}
    \centering
    \circicon{uublue}{\faSearch}

    \vspace{0.3cm}
    \textbf{Evaluate signals}

    \vspace{0.1cm}
    {\small generated by prediction models.}
  \end{columns}

\end{frame}

\begin{frame}{Groups \& Competition}
  \vspace{0.5cm}

  \begin{columns}[T]
    \column{0.49\textwidth}
    \centering
    \circicon{uugreen}{\faUsers}
    \vspace{0.3cm}

    \hspace*{-0.6em}\textbf{\large Groups}

    \vspace{0.3cm}

    \begin{itemize}
      \item Form groups of \textbf{3-4 students}
      \item Sign up via: \href{https://solisservices-my.sharepoint.com/:x:/g/personal/m_b_b_nielsen_uu_nl/IQDY7mAP4SzeSYspnBAQEB4uAbVFKlZWDADDHX-TFFCb3MI?e=Jkmd6r}{\textcolor{uublue}{Group-Signup-Sheet}}
      \item Hand-in assignment on Brightspace
      \item \textbf{Deadline:}
      \item[] \textbf{ Wednesday, March 11, 23:59}
    \end{itemize}

    \column{0.49\textwidth}
    \centering
    \circicon{uuyellow}{\faTrophy}
    \vspace{0.3cm}

    \textbf{\large Trading Competition}

    \vspace{0.3cm}

    \begin{itemize}
      \item Date: \textbf{March 26}
      \item Prizes (grade bonus for group assignment$^*$):

        \vspace{0.1cm}
        \hspace*{1.0em}%
        \begin{tabular}{@{}l@{\hspace{0.3em}}r@{}}
          \textcolor{uupurple}{\faAward}\ 1st: & +1.0\\
          \textcolor{uupurple}{\faAward}\ 2nd: & +0.6\\
          \textcolor{uupurple}{\faAward}\ 3rd: & +0.2
        \end{tabular}
    \end{itemize}
    \vspace{0.5em}
    {\raggedright
    \hrule width 2in
    \vspace{0.2em}
    \footnotesize $^*$ Group assignment grade is capped at 10.\par}
  \end{columns}
\end{frame}

% =============================================================================
% SECTION 2: Course overview
% =============================================================================
\sectiondivider{purple_pink.png}{Course Overview}

\begin{frame}{Trading Algorithms: Two Tracks}
  \vspace{0.5cm}

  \begin{center}
  \begin{tikzpicture}
    % Trading on the left (larger)
    \node[circle, draw=uuyellow, line width=1.5pt, minimum size=2.2cm, fill=uuyellow!10, drop shadow={opacity=0.15}] (trading) at (0,0) {};
    \node[font=\Large, color=uuyellow, text width=1.5em, align=center] at (trading.center) {\faChartBar};
    \node[below=0.15cm of trading] {\textbf{Trading}};

    % Signal track (top right, smaller)
    \node[circle, draw=uugreen, line width=1.5pt, minimum size=1.5cm, fill=uugreen!10, drop shadow={opacity=0.15}] (signal) at (5,2) {};
    \node[font=\normalsize, color=uugreen, text width=1.5em, align=center] at (signal.center) {\faLightbulb};
    \node[right=0.3cm of signal, align=left] {\textbf{Signal}\\[0.05cm]{\small Predictive Algorithms}\\{\small\color{darkbg!70} When to trade?}};

    % Execution track (bottom right, smaller)
    \node[circle, draw=uublue, line width=1.5pt, minimum size=1.5cm, fill=uublue!10, drop shadow={opacity=0.15}] (exec) at (5,-2) {};
    \node[font=\normalsize, color=uublue, text width=1.5em, align=center] at (exec.center) {\faCogs};
    \node[right=0.3cm of exec, align=left] {\textbf{Execution}\\[0.05cm]{\small Execution Algorithms}\\{\small\color{darkbg!70} How to trade?}};

    % Arrows from Trading to Signal and Execution (from different points on Trading)
    \draw[->, line width=2pt, darkbg] ([xshift=0.5cm, yshift=0.5cm]trading.east) -- ([xshift=-0.8cm, yshift= -0.3cm]signal.west);
    \draw[->, line width=2pt, darkbg] ([xshift=0.5cm, yshift=-0.5cm]trading.east) -- ([xshift=-0.8cm, yshift=0.3cm]exec.west);

    % Grey bidirectional arrow between Signal and Execution (same color as course plan)
    \draw[<->, line width=2pt, darkbg!20] ([yshift= -0.8cm]signal.south) -- ([yshift=0.8cm]exec.north);
  \end{tikzpicture}
  \end{center}
\end{frame}

\begin{frame}{Course Plan}
  \vspace{0.6cm}

  \centering
  \begin{tikzpicture}
    % Small circle icon styles are defined in the preamble

    % Legend below everything (centered)
    \node[anchor=center, font=\footnotesize, text=darkbg!60] at (4.5,-6.2) {\faLaptop\ = Workshop};

    % Topic 1 - Volatility (Signal)
    \node[smallicon=uugreen] (t1) at (1,0) {\color{uugreen}\faChartArea};
    \node[topicbox] at (2.1,0.15) {\textbf{Volatility Modelling}};
    \node[topicbox, font=\small, text=darkbg!70] at (2.1,-0.25) {\faLaptop\hspace{0.15cm}Forecasting volatility};
    %\node[anchor=west, font=\small, text=darkbg!60] at (8,0){\faLightbulb};
   \node[anchor=center, font=\small, text=darkbg!60] at (8.3,0) {\faLightbulb\hspace{0.1cm}\faCogs};

    % Topic 2 - Order Book (Execution)
    \node[smallicon=uublue] (t2) at (1,-1.3) {\color{uublue}\faBookOpen};
    \node[topicbox] at (2.1,-1.15) {\textbf{Order Book}};
    \node[topicbox, font=\small, text=darkbg!70] at (2.1,-1.55) {\faLaptop\hspace{0.15cm}Dual listing arbitrage};
    \node[anchor=west, font=\small, text=darkbg!60] at (8,-1.3) {\faCogs};

    % Topic 3 - Options (Signal + Execution)
    \node[smallicon=uupurple] (t3) at (1,-2.6) {\color{uupurple}\faBalanceScale};
    \node[topicbox] at (2.1,-2.45) {\textbf{Options (Pricing \& Trading)}};
    \node[topicbox, font=\small, text=darkbg!70] at (2.1,-2.85) {\faLaptop\hspace{0.15cm}Options Market Making};
    \node[anchor=center, font=\small, text=darkbg!60] at (8.3,-2.6) {\faLightbulb\hspace{0.1cm}\faCogs};

    % Topic 4 - Machine Learning (Signal)
    \node[smallicon=uuorange] (t4) at (1,-3.9) {\color{uuorange}\faBrain};
    \node[topicbox] at (2.1,-3.75) {\textbf{Machine Learning (ML)}};
    \node[topicbox, font=\small, text=darkbg!70] at (2.1,-4.15) {\faLaptop\hspace{0.15cm}Applying ML to financial data};
    \node[anchor=west, font=\small, text=darkbg!60] at (8,-3.9) {\faLightbulb};

    % Topic 5 - GenAI (Signal)
    \node[smallicon=uured] (t5) at (1,-5.2) {\color{uured}\faRobot};
    \node[topicbox] at (2.1,-5.05) {\textbf{Text-to-Data}};
    \node[topicbox, font=\small, text=darkbg!70] at (2.1,-5.45) {\faLaptop\hspace{0.15cm}Automated text analysis};
    \node[anchor=west, font=\small, text=darkbg!60] at (8,-5.2) {\faLightbulb};

    % Connecting line
    \draw[darkbg!20, line width=2pt] (t1.south) -- (t2.north);
    \draw[darkbg!20, line width=2pt] (t2.south) -- (t3.north);
    \draw[darkbg!20, line width=2pt] (t3.south) -- (t4.north);
    \draw[darkbg!20, line width=2pt] (t4.south) -- (t5.north);

  \end{tikzpicture}
\end{frame}

% =============================================================================
% ASSESSMENT (no section divider)
% =============================================================================
\begin{frame}{Assessment Overview}
  \vspace{1.2cm}

  \begin{center}
  \begin{tikzpicture}
    % Midterm (left)
    \node (mid) at (0,0) {\circicon{uugreen}{\faEdit}};
    \node[below=0.3cm of mid, align=center] (midlabel) {\textbf{Midterm} {\small (10\%)}\\[0.15cm]{\small Order Book}};

    % Assignment (center)
    \node (assign) at (5,0) {\circicon{uupurple}{\faLaptopCode}};
    \node[below=0.3cm of assign, align=center] (assignlabel) {\textbf{Assignment} {\small (30\%)}\\[0.15cm]{\small Options Market Making}};

    % Final (right)
    \node (final) at (10,0) {\circicon{uublue}{\faGraduationCap}};
    \node[below=0.3cm of final, align=center] (finallabel) {\textbf{Final Exam} {\small (60\%)}\\[0.15cm]{\small \only<1>{\faQuestion}\only<2->{All topics possible}}\\[0.15cm]{\only<2->{except Order Book}}};

    % Icons aligned vertically at same y position
    \node[font=\small, text=darkbg!60] at (0,-3.8) {\faCogs};
    \node[font=\small, text=darkbg!60] at (5,-3.8) {\faCogs};
    \node[font=\small, text=darkbg!60] at (10,-3.8) {\faLightbulb\hspace{0.1cm}\faCogs};

    % Arrows
    %\draw[-, line width=2pt, darkbg!30] ([xshift=1.1cm]mid.east) -- ([xshift=-1.1cm]assign.west);
    %\draw[-, line width=2pt, darkbg!30] ([xshift=1.1cm]assign.east) -- ([xshift=-1.1cm]final.west);
  \end{tikzpicture}
  \end{center}
\end{frame}

% =============================================================================
% SECTION 4: AI IN SOCIETY
% =============================================================================
\sectiondivider{blue_turq.png}{AI in society (and this course)}

\begin{frame}{Two Faces of AI}
  \vspace{0.5cm}

  \begin{columns}[T]
    \column{0.45\textwidth}
    \centering
    \circicon{uublue}{\faCog}
    \vspace{0.3cm}

    \textbf{\large Automation}

    \vspace{0.3cm}

    \begin{itemize}
      \item Replace human tasks
      \item Substitute capital for labor
      \item Free up time
    \end{itemize}

    \column{0.45\textwidth}
    \centering
    \circicon{uupurple}{\faHandsHelping}
    \vspace{0.3cm}

    \textbf{\large Augmentation}

    \vspace{0.3cm}

    \begin{itemize}
      \item Extend capabilities
      \item Higher labor productivity
      \item Enable new expertise
    \end{itemize}
  \end{columns}

  \vspace{0.6cm}

  \begin{center}
    {\color{darkbg!70}\textit{Author (2024): The impact depends on how we choose to deploy AI}}
  \end{center}
\end{frame}

\begin{frame}{Autor (2024) Applying AI to Rebuild Middle Class Jobs}
  \vspace{0.3cm}

  \textbf{Historical arc of expertise:}
  \begin{itemize}
    \item \textbf{Artisanal} (pre-Industrial): Skilled craftsmen with procedural + judgment expertise
    \item \textbf{Mass expertise} (Industrial): Factory workers following rules, less judgment
    \item \textbf{Elite expertise} (Computer Age): Decision-making concentrated among college-educated professionals
  \end{itemize}

  \vspace{0.3cm}

  \textbf{AI's unique opportunity:}
  \begin{itemize}
    \item AI can \textbf{extend expertise} to more workers
    \item Complements judgment rather than just automating procedures
    \item Could rebuild middle-skill, middle-class jobs
  \end{itemize}
\end{frame}

\begin{frame}{Mollick et al. (2024) Navigating the Jagged Technological Frontier}
  \vspace{0.5cm}

  \begin{columns}[T]
    \column{0.55\textwidth}

    AI capabilities are \textbf{uneven} across tasks:

    \vspace{0.3cm}

    \begin{itemize}
      \item Excellent at some tasks, poor at others 
      \item Irregular boundary, ``jagged frontier''
    \end{itemize}

    \vspace{0.4cm}

    \textbf{What remains essential:}
    \begin{itemize}
      \item Critical thinking
      \item Domain expertise
      \item Judgment for high-stakes decisions
    \end{itemize}

    \column{0.4\textwidth}
    \centering
    \vspace{0.3cm}

    \includegraphics[width=\textwidth]{\uuaifassets/graphs/Jagged-frontier-blog-drawing-of-frontier-transparent.png}
  \end{columns}
\end{frame}

\begin{frame}{The Concern: Automation vs Augmentation}
  \vspace{0.5cm}

  \begin{center}
    \begin{tcolorbox}[
      colback=softcoral,
      colframe=uured,
      arc=4mm,
      boxrule=1.5pt,
      width=0.85\textwidth,
      halign=center,
      top=0.4cm,
      bottom=0.4cm
    ]
      {\large \textbf{Too early automation hurts augmentation}}
    \end{tcolorbox}
  \end{center}

  \vspace{0.4cm}

  \textbf{Learning by doing still matters:}
  \begin{itemize}
    %\item Junior vs Senior roles in investment banking
    \item Expertise built through practice (struggle!)
    \item AI tools require training to use effectively
  \end{itemize}

  \vspace{0.3cm}

  \textbf{The risk:}
  \begin{itemize}
    \item Skipping foundational learning $\rightarrow$ never develop expert judgment
    \item Tools are less useful (perhaps dangerous) without it
  \end{itemize}
\end{frame}




% Add more slides here
{\renewcommand{\thefootnote}{\fnsymbol{footnote}}%
\statemphasisgreen{Researcher at Anthropic found\footnote{Anthropic is the company behind Large Language Model Claude.}}{34\%}{increase in average test score from preparing \emph{without} AI-assistance.}}

\begin{frame}{Shen \& Tamkin (2026) How AI Impacts Skill Formation}
  \centering
  \vspace{0.3cm}
  \includegraphics[width=\textwidth, height=0.85\textheight, keepaspectratio]{\uuaifassets/graphs/Anthropic - effect of AI on coding skills.jpeg}

  \vspace{0.2cm}
  {\footnotesize From \href{https://www.anthropic.com/research/AI-assistance-coding-skills}{\textcolor{uublue}{How AI assistance impacts the formation of coding skills}}.}
\end{frame}

% =============================================================================
% AI IN THE COURSE
% =============================================================================
\begin{frame}[nologo]{AI in this Course}
  \vspace{0.3cm}

  \begin{center}
  \begin{tcolorbox}[
    colback=softyellow,
    colframe=uuyellow,
    arc=4mm,
    boxrule=1.5pt,
    width=0.95\textwidth,
    title={\textbf{Level 3: AI-Assisted Editing of Student-Generated Content}},
    fonttitle=\bfseries,
    coltitle=darkbg,
    colbacktitle=uuyellow
  ]
    \textbf{What's allowed:}
    \begin{itemize}
      \item Use AI to improve clarity or quality of \emph{your own} work
      \item AI can be used, but you must complement your work with an \textbf{AI prompt log}
    \end{itemize}

    \vspace{0.2cm}

    \textbf{What's required:}
    \begin{itemize}
      \item Include an \textbf{AI disclosure statement} explaining how AI was used
      \item Keep a record of your AI prompt logs in case we request them
    \end{itemize}

    \vspace{0.2cm}

    \textbf{What's NOT allowed:}
    \begin{itemize}
      \item AI cannot be used to generate complete assignments
    \end{itemize}
  \end{tcolorbox}
  \end{center}

  \vspace{0.3cm}
  {\small\color{darkbg!70} More info: \texttt{uu.nl/en/organisation/ai-policy/students/ai-index}}
\end{frame}

% =============================================================================
% CLOSING SLIDE
% =============================================================================
\closingslide{Questions?}{endpage_light_green.png}

\end{document}
