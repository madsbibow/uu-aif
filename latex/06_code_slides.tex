% 06_code_slides.tex - Code slides with verbatim/listings
% Compile from the uu-aif directory
\documentclass[aspectratio=169]{beamer}
\usepackage{uu-aif-beamer}

% Listings package for code
\usepackage{listings}

% Define Python style matching UU theme
\lstdefinestyle{uupython}{
  language=Python,
  basicstyle=\ttfamily\small\color{uuwhite},
  keywordstyle=\color{uuyellow},
  commentstyle=\color{uugreen},
  stringstyle=\color{softcoral},
  numberstyle=\tiny\color{gray},
  backgroundcolor=\color{darkbg},
  frame=none,
  showstringspaces=false,
  showspaces=false,
  showtabs=false,
  breaklines=true,
  breakatwhitespace=true,
  tabsize=4,
  xleftmargin=0pt,
  xrightmargin=0pt,
  aboveskip=0pt,
  belowskip=0pt,
  morekeywords={def, return, import, from, as, if, else, elif, for, while, True, False, None, print, self},
  literate={*}{{\textcolor{uuwhite}{*}}}1
           {=}{{\textcolor{uuwhite}{=}}}1
           {-}{{\textcolor{uuwhite}{-}}}1
           {+}{{\textcolor{uuwhite}{+}}}1
           {/}{{\textcolor{uuwhite}{/}}}1
           {(}{{\textcolor{uuwhite}{(}}}1
           {)}{{\textcolor{uuwhite}{)}}}1
           {[}{{\textcolor{uuwhite}{[}}}1
           {]}{{\textcolor{uuwhite}{]}}}1
           {:}{{\textcolor{uuwhite}{:}}}1
           {,}{{\textcolor{uuwhite}{,}}}1
           {.}{{\textcolor{uuwhite}{.}}}1
}

% Set default style
\lstset{style=uupython}

% Style for inline code (on cream background)
\lstdefinestyle{uupython-inline}{
  language=Python,
  basicstyle=\ttfamily\color{darkbg},
  keywordstyle=\color{uupurple},
  commentstyle=\color{uugreen},
  stringstyle=\color{uured},
  showstringspaces=false,
  morekeywords={def, return, import, from, as, if, else, elif, for, while, True, False, None, print, self},
}

% Convenience command for inline code
\newcommand{\pycode}[1]{\lstinline[style=uupython-inline]{#1}}

\begin{document}

\titleslide{Code Slides}{Python Implementation}{January 2026}

% Simple code block
\begin{frame}[fragile]{Python Implementation}

  \begin{codebox}
  \begin{lstlisting}
def black_scholes_call(S, K, T, r, sigma):
    # Calculate European call option price
    d1 = (np.log(S/K) + (r + 0.5*sigma**2)*T) \
         / (sigma * np.sqrt(T))
    d2 = d1 - sigma * np.sqrt(T)
    return S*norm.cdf(d1) - K*np.exp(-r*T)*norm.cdf(d2)
  \end{lstlisting}
  \end{codebox}

\end{frame}

% Code with explanation
\begin{frame}[fragile]{Delta Calculation}

  \begin{columns}
    \column{0.55\textwidth}
    \begin{codebox}
    \begin{lstlisting}
def delta(S, K, T, r, sigma):
    d1 = calculate_d1(...)
    return norm.cdf(d1)
    \end{lstlisting}
    \end{codebox}

    \column{0.42\textwidth}
    \begin{itemize}
      \item Delta = $N(d_1)$
      \item Range: $[0, 1]$ for calls
      \item Measures price sensitivity
    \end{itemize}
  \end{columns}

\end{frame}

% Multiple code snippets
\begin{frame}[fragile]{All Greeks}

  \begin{columns}[T]
    \column{0.48\textwidth}
    \begin{codebox}
    \begin{lstlisting}
# Delta
delta = norm.cdf(d1)

# Gamma
gamma = norm.pdf(d1) / \
    (S * sigma * sqrt(T))
    \end{lstlisting}
    \end{codebox}

    \column{0.48\textwidth}
    \begin{codebox}
    \begin{lstlisting}
# Vega
vega = S * sqrt(T) * norm.pdf(d1)

# Theta
theta = -S * norm.pdf(d1) * \
    sigma / (2 * sqrt(T))
    \end{lstlisting}
    \end{codebox}
  \end{columns}

\end{frame}

% Usage example
\begin{frame}[fragile]{Example Usage}

  \begin{codebox}
  \begin{lstlisting}
# Example: Price a call option
S = 100      # Stock price
K = 105      # Strike
T = 0.5      # Time (years)
r = 0.05     # Risk-free rate
sigma = 0.2  # Volatility

price = black_scholes_call(S, K, T, r, sigma)
print(f"Call price: ${price:.2f}")
# Output: Call price: $5.57
  \end{lstlisting}
  \end{codebox}

\end{frame}

% Inline code example
\begin{frame}[fragile]{Inline Code}

  You can also use inline code like \pycode{import numpy as np} within text.

  \vspace{1em}

  The function \pycode{black_scholes_call(S, K, T, r, sigma)} takes five parameters:

  \begin{itemize}
    \item \pycode{S} -- Current stock price
    \item \pycode{K} -- Strike price
    \item \pycode{T} -- Time to expiration (in years)
    \item \pycode{r} -- Risk-free interest rate
    \item \pycode{sigma} -- Volatility
  \end{itemize}

\end{frame}

\closingslide{Questions?}{endpage_light_green.png}

\end{document}
